% Tipo di documento. L'uso di twoside implica che i capitoli inizino sempre con la prima pagina a sinistra, eventualmente lasciando una pagina vuota nel capitolo precedente. Se questa cosa è fastidiosa, è possibile rimuoverlo. 
\documentclass[a4paper,openright]{report}

% Dimensione dei margini
\usepackage[a4paper,top=3cm,bottom=3cm,left=3cm,right=3cm]{geometry} 
% Dimensione del font
\usepackage[fontsize=12pt]{scrextend}
% Lingua del testo
\usepackage[italian,english]{babel}
% Lingua per la bibliografia
% \usepackage[fixlanguage]{babelbib}
% Codifica del testo
\usepackage[utf8]{inputenc} 
% Font mono (quello di default non supporta il grassetto)
\usepackage{courier}
% Encoding del testo
\usepackage[T1]{fontenc}
% Lorem Ipsum
\usepackage{lipsum}
% Bibliografia
\usepackage[
backend=biber,
bibstyle=other/custom-numeric,
citestyle=numeric,
sorting=none,
]{biblatex}
% Sorgente della bibliografia
\addbibresource{chapters/Bibliography.bib}
% Per ruotare le immagini
\usepackage{rotating}
% Per cambiare i capitoli
\usepackage{titlesec}
\titleformat{\chapter}[hang]
  {\normalfont\bfseries\huge}{\thechapter}{1em}{\huge}
% Per mostrare nell'indice anche le subsubsection
\setcounter{tocdepth}{3}
% Per modificare l'header delle pagine 
\usepackage{fancyhdr}
% Librerie matematiche
\usepackage{amssymb}
\usepackage{amsmath}
\usepackage{amsthm}         
% Uso delle immagini
\usepackage{graphicx}
% Uso dei colori
\usepackage[dvipsnames,svgnames,x11names]{xcolor}
% Uso dei listing per il codice
\usepackage{listings}          
% Per inserire gli hyperlinks tra i vari elementi del testo 
% \usepackage[hang, flushmargin]{footmisc}
% Diversi tipi di sottolineature
\usepackage[normalem]{ulem}
% Fix relative indenting
\usepackage{lstautogobble}
% Code coloring
\usepackage{color}
% Nice font
\usepackage{zi4}
% Loads the Latin Modern font
\usepackage{lmodern}
% For text alignment
\usepackage{ragged2e}
% Customizes the appearance of figure and table captions
\usepackage{caption}
% Code highlighting
\usepackage[newfloat,outputdir=out]{minted}
% Better frame for minted
\usepackage{tcolorbox}
\tcbuselibrary{minted,skins,breakable,xparse}
\tcbuselibrary{listingsutf8} % Allows minted in tcolorbox
% Additional options to control the placement of figures and tables
\usepackage{float}
% No paragraph indentation instead vspaces
\usepackage[skip=5pt]{parskip}
% Allows customization of line spacing
\usepackage{setspace}
% Enables switching or substituting hyphenation patterns in your document
\usepackage{hyphsubst}
% Improves the overall typography of the document by subtly adjusting spacing and font shapes
\usepackage{microtype}
% Allows the creation of hyperlinks in the document
\usepackage{hyperref}
% Adds back-references from footnotes to the point in the text where the footnote was used
\usepackage{footnotebackref}
% Tools for formatting and managing quotes in your document
% To be used after minted
\usepackage{csquotes}
% Adding sub-captions to figures and tables
\usepackage{subcaption}
% Drawing electrical and electronic circuit diagrams
\usepackage{circuitikz}
% Permits to patch macros
\usepackage{xpatch}


% Comments ( substitute command name and name )
\newcommand{\cuno}[1]{\textcolor{red}{[\bfseries Persona 1: #1]}}
\newcommand{\cdue}[1]{\textcolor{red}{[\bfseries Persona 2: #1]}}

% Don't include in \leftmark "Chapter"
\renewcommand{\chaptermark}[1]{%
  \markboth{\thechapter. #1}{}%
}
% Modifica lo stile dell'header
\pagestyle{fancy}
\fancyhf{}
% These work if the document is in twoside mode
% \fancyhead[CE,CO]{\rightmark}
% \fancyhead[LE,RO]{\textbf{\thepage}}

% Make the header
% on the left side, show the chapter title if no section is present, otherwise the section title
% on the right side, show the page number
\fancyhead[L]{
  \ifnum\value{section}=0 % Check if no section exists
    \leftmark            % Use Chapter title
  \else
    \rightmark           % Use Section title
  \fi
}
\fancyhead[R]{\textbf{\thepage}}
\fancyfoot{}
\setlength{\headheight}{17pt}

% Rimuove il numero di pagina all'inizio dei capitoli
\fancypagestyle{plain}{
  \fancyfoot{}
  \fancyhead{}
  \renewcommand{\headrulewidth}{0pt}
}

% Custom colors
\definecolor{DarkGreen}{RGB}{20,80,40}
\definecolor{Unipi}{RGB}{00,85,143} 
\definecolor{term_bg_color}{RGB}{240,241,240} 

% Modifica dello stile dei riferimenti
\hypersetup{
    colorlinks,
    linkcolor=Unipi,
    citecolor=Unipi,
    urlcolor=Unipi
}

% Interlinea
\setstretch{1.1}

% Supercite tipo wikipedia
\DeclareCiteCommand{\supercite}[\mkbibsuperscript]
  {\iffieldundef{prenote}
     {}
     {\BibliographyWarning{Ignoring prenote argument}}%
   \iffieldundef{postnote}
     {}
     {\BibliographyWarning{Ignoring postnote argument}}}
  {\usebibmacro{citeindex}%
   \bibopenbracket\usebibmacro{cite}\bibclosebracket}
  {\supercitedelim}
  {}


% Aggiunti definizioni, teoremi, linea e listing
\newtheorem{definition}{Definizione}[section]
\newtheorem{theorem}{Teorema}[section]
\providecommand*\definitionautorefname{Definizione}
\providecommand*\theoremautorefname{Teorema}
\providecommand*{\listingautorefname}{Listing}
\providecommand*\lstnumberautorefname{Linea}

\newcommand{\cboh}[1]{{\textcolor{blue}[\textcolor{magenta}{\bf{!!: }}{ \textcolor{blue}{#1]}}}}

% LaTeX stops trying to balance the text across pages, allowing pages to have uneven bottom margins
\raggedbottom

%----------------------------------------------------

\begin{document}

\setlength{\fboxsep}{0pt}
\selectlanguage{english}

% \hyphenpenalty=10000
% \exhyphenpenalty=10000

% \loadspellchecklist[it][wordlist.txt]
% \setupspellchecking[state=start]


% Define a custom style for your tcolorbox
\tcbset{
    listing engine=minted,
    minted options={
            fontsize=\small,
            linenos,
            numbersep=4mm,
            breaklines,
            baselinestretch=0.9
        },
    colback=term_bg_color,
    colframe=Unipi,
    fonttitle=\bfseries,
    listing only,
    left=1.5mm,
    arc=0.5mm,
    enhanced,
    breakable,
    before skip=\baselineskip,
    grow to left by=2mm,grow to right by=2mm,
    enlarge bottom by=-8pt
}

% Define command to include a minted file
\newtcbinputlisting{\mintedCode}[3][]{%
    listing file={#3},
    minted language={#2}
}

% Space between caption and figure, listing
\setlength{\abovecaptionskip}{8pt}
\setlength{\belowcaptionskip}{6pt}
\captionsetup[figure]{belowskip=-12pt}

% Define a new environment for code listings
\newenvironment{code}
{\captionsetup{type=listing}}
{\par\noindent\ignorespacesafterend}

\SetupFloatingEnvironment{listing}{name=Listing, listname=List of Listings}

\usemintedstyle{sas}
% \usemintedstyle{unipi}
% \usemintedstyle{xcode}
% \usemintedstyle[output]{rrt}


\begin{titlepage}
\begin{figure}[!htb]
    \centering
    \includegraphics[keepaspectratio=true,scale=0.5]{images/Frontespizio/cherubinFrontespizio.eps}
\end{figure}

\begin{center}
    \LARGE{UNIVERSITY OF PISA}
    \vspace{5mm}
    \\ \large{MSc in Computer Engineering}
    \vspace{5mm}
    \\ \LARGE{Project for Performance Evaluation of Computer Systems and Networks}
\end{center}

\vspace{15mm}
\begin{center}
    {\LARGE{\bf Multi-core scheduling}}
    
    % Se il titolo è abbastanza corto da stare su una riga, si può usare
    
    % {\LARGE{\bf Un fantastico titolo per la mia tesi!}}
\end{center}
\vspace{30mm}

\begin{minipage}[t]{0.47\textwidth}
	{\large{Professors:}{\normalsize\vspace{3mm}\bf\\ \large{Prof. Giovanni Stea}}{\normalsize\vspace{3mm}
               \bf\\ \large{Ing. Giovanni Nardini }}}

\end{minipage}
\hfill
\begin{minipage}[t]{0.51\textwidth}\raggedleft
	{\large{Students:}
            {\normalsize\vspace{3mm} \bf\\ \large{Taulant Arapi (645308)}}
            {\normalsize\vspace{3mm} \bf\\ \large{Francesco Barcherini (645413)}}
            {\normalsize\vspace{3mm} \bf\\ \large{Antonio Ciociola (645324)}}
        }
            
\end{minipage}

\vspace{30mm}
\hrulefill
\\\centering{\large{ACADEMIC YEAR 2024/2025}}

\end{titlepage}

% \begin{titlepage}
% \begin{figure}[!htb]

% \begin{center}
% {
%     \includegraphics[keepaspectratio=true,scale=0.5]{images/Frontespizio/cherubinFrontespizio.eps}
% }
% \end{center}

% \end{figure}

% \begin{center}
%     \LARGE{UNIVERSITÀ DI PISA}
% \end{center}

% \vspace{15mm}
% \begin{center}
%     {\LARGE{\bf
%     TITOLO  
%     \\ \vspace{3mm}
%     TITOLO RIGA 2
%     }}
% \end{center}

% \vspace{50mm}


% \begin{minipage}[t]{0.47\textwidth}
% 	{
%         \large{Authors:}
%         {\normalsize\vspace{3mm}\bf\\ 
%         \large{Taulant Arapi}
%         \normalsize\vspace{3mm}\bf \\
%         \large{Francesco Barcherini}
%         \normalsize\vspace{3mm}\bf \\
%         \large{Antonio Ciociola}
%         }
%     }
% \end{minipage}


% \vfill
% \hrulefill
% \\\centering{\large{ANNO ACCADEMICO 2024/2025}}

% \end{titlepage}
\stepcounter{page}

\tableofcontents


\chapter{Photos}

\includegraphics[width=0.6\textwidth]{images/example/pag1.jpeg}
\includegraphics[width=0.6\textwidth]{images/example/pag2.jpeg}
\chapter{Introduction}

\section{Description and specification}
It is required to analyze a multi-core computer equipped with \( N \) CPUs, which execute multiple interactive processes. The processes are dynamically generated at intervals of \( T \) seconds. Each process has a total duration \( D \), which is divided into three distinct execution phases:
\begin{enumerate}
    \item initial processing phase;
    \item I/O operation phase;
    \item final processing phase.
\end{enumerate}

The times \( T \) and \( D \) are defined as IID random variables. The processes are categorized into two types and generated as \textit{CPU bound} with probability \( p \) or \textit{I/O bound} with probability \( 1 - p \). The difference is that:
\begin{itemize}
    \item in CPU-bound processes the I/O operation phase constitutes 20\% of the total duration \( D \), and the other two phases both the 40\%;
    \item in I/O-bound processes the I/O operation phase constitutes 80\% of their total duration \( D \), and the other two phases both the 10\%;
\end{itemize}

Processes are assigned to CPUs by the operating system’s scheduler, which selects tasks from a list of “ready” processes. When a CPU becomes idle for entering the I/O operation phase or ending the final processing phase of a process, the scheduler immediately assigns it to a new ready process, if present. Once a process completes its I/O operation phase, it is marked as ready again and is returned to the scheduling queue. A process exits the system once it has successfully completed all three phases of execution.

The scheduler can be implemented to follow two distinct policies:
\begin{itemize}
    \item First Come First Served (FCFS): processes are scheduled in the order of their arrival after the generation or after the end of the I/O phase;
    \item Shortest Job First (SJF): scheduling is determined based on the shortest remaining time until either the process’s I/O phase or its completion.
\end{itemize}

\section{Objectives}

The aim of this project is to analyze the execution of a multi-core 
scheduling system under varying operational scenarios. 

One of the the objectives is 
to determine the optimal characteristics of the system in order to make 
it work without the need to discard processes. Furthermore, the analysis 
aims to assess the impact of the number of CPUs and of the distributions
of generation and duration times as regards the system’s performance.
Another objective is to realize a resource efficiency evaluation. 

In particular, 
the different configurations taken into 
consideration are the scheduling policies, the number of CPUs in the 
system, the type of the processes (CPU bound or I/O bound), the 
generation intervals and the duration of processes.

The overarching goal is to generate insights about the impact of the scheduling 
strategy and of the functional parameters of the system to evaluate
the performance of multi-core environments, 
enhancing system efficiency and minimizing delays.


\section{Indices}

To evaluate the system’s performance comprehensively, the following indices will be analyzed in detail:
\begin{itemize}
    \item turnaround time $R$: defined as the total time elapsed from the arrival of a process in the system to its completion. This metric provides a measure of the computer’s overall responsiveness and of the time required for different types of processes to be executed;
    \item waiting time $W$: the cumulative time a process spends in the ready queue before being assigned to a CPU for execution. This metric represents the impact on the turnaround time of the queuing policy and of the workload of the system;
    \item CPU utilization of the $n_{th}$ CPU $\rho_n$: the fraction of time that each CPU is actively engaged in executing processes instead of being idle. This metric reflects the efficiency of resource usage within the system;
    \item active CPUs over time $N_A$: a dynamic measure of the number of CPUs actively processing tasks at any given moment. This index helps assess load distribution and system evolution;
    \item queue length over time $N_q$: tracks the size of the ready queue throughout the simulation, highlighting bottlenecks and variations in process scheduling.
\end{itemize}

These indices will be statistically analyzed to identify trends, anomalies, edge cases and key factors influencing system performance. By examining these metrics under diverse configurations of \( N \), \( p \) and scheduler policies, the project aims to provide actionable recommendations for improving the efficiency and adaptability of multi-core scheduling systems.

\section{Assumptions}
To perform the analysis, the following assumptions are made:
\begin{itemize}
    \item there is sufficient memory to store the list of processes ready for execution, so the ready queue will be considered as infinite and there is no need to discard processes;
    \item there is no overhead nor delay for every process from the generation to the arrival in the ready queue;
    \item there is no overhead nor delay for every process in the transitions between the ready queue, the CPUs and the I/O phase;
    \item the only workload of the CPUs is the execution of the processes. Scheduling, I/O, process generation do not require CPU time;
    \item the overhead for the sorting algorithm in SJF policy and for the queue handling is negligible;
    \item with the shortest job first (SJF) scheduling, an accurate estimate of the execution times is known a priori. If the scheduler is FCFS, this data is not used;
    \item the generation time, the duration time and the characteristic of a process of being CPU-bound or I/O-bound are respectively IID Random Variables;
    \item I/O operations and CPU processing do not interfere with each other.
\end{itemize}




\section{Preliminary stability estimation}

The system can be modeled as a queueing network as shown in \autoref{fig:schema}.

\begin{figure}[h]
    \captionsetup{type=figure}
    \centering
    \includegraphics[width=0.6\textwidth]{images/03/schema.pdf}
    \captionof{figure}{Model of the system as a queueing network}
    \label{fig:schema}
\end{figure}

Let's analyze the system when the process generation time $T$ and the process duration $D$ are exponentially distributed, with mean values $E[T]$ and $E[D]$ respectively, and when the queues follow a FCFS scheduling policy.
New processes are generated at rate $\gamma_1=\frac{1}{E[T]}$.
The initial processing phase and the final processing phase are modeled by the $M/M/C$ Service Center $1$, where $C=N$ is the number of CPUs. 
Here it is assumed that the service time $t_{cpu}$ can be $0.4D$ for CPU bound jobs and $0.1D$ for I/O bound jobs. This means that, for total probability theorem, the mean value of the service time of the CPU processing is 
\[
E[t_{cpu}]=0.4 E[D] \cdot p+0.1E[D]\cdot (1-p) = 0.3E[D]\cdot p+0.1E[D]
\]
so that 
\[
\mu_{cpu}=\frac{1}{E[t_{cpu}]}=\frac{1}{0.3E[D]\cdot p+0.1E[D]}
\]
After the CPU processing phase, we can assume that it is equally likely that the process is at the beginning of the I/O phase or at the end of final processing phase. In the second case, it leaves the system with routing probability $\pi_0=\frac{1}{2}$.
If the process starts the I/O phase, it is routed with probability $\pi_1=\frac{1}{2}$ to the $M/M/\infty$ Delay Center $2$ with service time $t_{IO}$. 
Reasoning as before, the mean value of the service time of the I/O processing is $0.2D$ for CPU bound jobs and $0.8D$ for I/O bound jobs. The mean value of the service time of the I/O processing phase is
\[
E[t_{IO}]=0.2E[D]\cdot p+0.8E[D]\cdot (1-p)=0.8E[D]-0.6E[D]\cdot p
\]
so that
\[
\mu_{IO}=\frac{1}{E[t_{IO}]}=\frac{1}{0.8E[D]-0.6E[D]\cdot p}
\]

Both service centers are of type $M/M/C$, the external arrival $\gamma_1$ is Poissonian, routing probabilities are state-independent and arcs are traversed in zero time.
Therefore the hypothesis of Jackson's theorem are satisfied and the system admits a product form if the stability conditions $\rho_i=\frac{\lambda_i}{C_i\cdot \mu_i}<1$ are satisfied for each service center $i$.
To compute $\lambda_1$ and $\lambda_2$ we can define the routing matrix $\Pi=\begin{bsmallmatrix} 0 & \frac{1}{2} \\ 1 & 0 \end{bsmallmatrix}$. Arrivals are $\gamma=\begin{bsmallmatrix} \gamma_1 \\ 0 \end{bsmallmatrix}$.
It is $\lambda=(I-\Pi^T)^{-1}\cdot \gamma$. We can observe that, since the input and output must balance, $\gamma_1=\frac{\lambda_1}{2}$ and from the routing we get $\lambda_2=\frac{\lambda_1}{2}$. At the end we get $\lambda_2=\gamma_1=\frac{1}{E[T]}$ and $\lambda_1=\frac{2}{E[T]}$.

The stability condition for the delay center is always true, so the system is stable if the first service center is stable.
To have $\rho_1<1$ we get $\frac{\lambda_1}{N\cdot \mu_{cpu}}<1$: substituting and rearranging we get
\begin{equation}
    N > \frac{3p+1}{5}\cdot \frac{E[D]}{E[T]} 
\end{equation}

The metrics to be computed can be obtained from the analysis of the two
service centers. The mean number of busy CPUs is, for example,
\begin{equation}
    E[N_{cpu}]=\frac{\lambda_1}{\mu_{cpu}}=\frac{3p+1}{5}\cdot \frac{E[D]}{E[T]}
\end{equation}

And, with some computation based on known formulas, it is possible to get a theoretical estimation of
the mean number of processes in the ready queue $E[N_q]=E[N_{q_1}]$, of the mean
waiting time in the ready queue $E[W] = 2E[W_1]$ and of the turnaround
time $E[R] = E[W]+E[D]$.

\chapter{Implementation}
We implemented the Computer system in OMNeT++.\\
Figure \autoref{fig:omnetpp_implementation} provides an overview of the system implementation. Below is a detailed explanation of each module:
\begin{itemize}
    \item \texttt{ProcessGenerator}: this module generates processes according to a specified interarrival time distribution. Each process is assigned a total duration $D$, and the duration of each phase (initial, I/O, and final) is set to $0.4 D$, $0.2 D$, and $0.4 D$ if the process is CPU bound, or $0.1 D$, $0.8 D$, and $0.1 D$ if the process is I/O bound. The process is then sent to the scheduler.
    \item \texttt{scheduler}: this module contains an infinite-capacity queue. If \texttt{isFCFS} is set to \texttt{true}, a FIFO queue is used. Otherwise, a priority queue is used, with processes ordered by increasing duration until the I/O phase or termination (shortest job first).\\
    When a new process arrives, the scheduler sends it to the first free CPU. If no CPU is free, the process is enqueued. When a CPU is freed, the next process in the queue is sent.\\
    When a process enters the I/O phase, the scheduler waits for the specified I/O phase duration and adds the process back into the queue.\\
    When a process terminates, its turnaround time and waiting time are emitted as signals.\\
    The scheduler also keeps track of the number of active CPUs and the queue length, emitting those every time a message is received.

    \item \texttt{cpu}: There are $N$ CPUs which operate independently. Each one, when it is free, receives a process from the scheduler and simulates its execution. When it finishes the execution phase (i.e. the I/O phase is reached or the process terminates), a message is sent to the scheduler, indicating that the CPU is now free. Every CPU records its utilization, emitting the busy time as a signal when the CPU is freed.
\end{itemize}

\begin{figure}[H]
    \captionsetup{type=figure}
    \centering
    \includegraphics[width=0.4\textwidth]{images/example/Computer.pdf}
    \captionof{figure}{Overview of the system implementation.}
    \label{fig:omnetpp_implementation}
\end{figure}

\chapter{Verification}

\section{Code verification}
The code has been verified to work as expected with the debug GUI that comes 
with OMNeT++. In particular, the code has been tested trying to execute all the 
possible paths of the code in order to simulate all different cases, and comparing 
the actual state of the system with the expected one.

For the analysis, the specific constructs of OMNeT++ (such as cQueue for the queues 
or user defined messages) allowed to better follow the flow of events in the simulations.
In fact the inspection of the content of messages and the state of the queues has been 
crucial to understand the behaviour of the system.

Additional information has been gathered printing log messages in the console with the 
use of \texttt{EV}, in order to have a better comprehension of the flow of the simulation.
For example, \texttt{EV} has been very useful to track the main events of the 
simulation, such as the arrival of a new process, the start and end of the three phases 
of execution, the assignment of a process to a CPU. 
A simple graphical representation of the processes queued waiting to be assigned to a 
CPU helped to verify the correct behaviour of the scheduler in both FCFS and SJF policies.

\section{Degeneracy test}
Degeneracy tests aim to verify the correct behaviour of the system at extreme 
values of the parameters. The main quantitative parameters are the number of CPUs 
\texttt{N}, the probability of a process to be CPU bound \texttt{p}, the mean generation time
\texttt{E[T]} and the mean process duration \texttt{E[D]}. The possible scenarios
can be tested with FCFS or SJF scheduling policy and with exponential or uniform 
distributions of the two times.

The probability \texttt{p} set at 0 or 1 leads to the generation of
only I/O bound or CPU bound processes. The simulation works as expected in the 
two cases with different values of the statistics (ex. turnaround and waiting time, 
number of busy processors etc.), without affecting the functionality of the system.

When the number of CPUs is set to 0, no CPU is instantiated and the processes
are never assigned to a CPU. The system works as expected, with the processes
waiting in the queue until the finish of the simulation time or the end of
memory availability. Tests with only 1 CPU show a good behaviour and no problem
related to the handling of vectors of gates with length 1. The test with
a huge number of CPUs (100) is not different from every test with more than 
one CPU with respect to the multicore functionalities of the computer; obviously,
in this case, there is always an available CPU to assign a process to and 
there is no queueing.

Talking about the mean generation time \texttt{E[T]} and the mean process duration 
\texttt{E[D]}, what matters is the ratio between the two. To simulate scenarios
with very large (small) generation time and very small (large) process duration, the two
parameters have been set to $10^3$ms and $10^{-3}$ms respectively, and vice versa.
In the case of $\texttt{E[T]}>>\texttt{E[D]}$, every process executes its three phases 
before the generation of a new one, so there is no queueing as expected.
In the case of $\texttt{E[T]}<<\texttt{E[D]}$, the processes are generated at a very high rate
so it is necessary to set \texttt{sim-time-limit} to a small value to 
end the simulation in a reasonable time. The system works as expected, with
a high number of processes in the queue and CPUs always busy.

\section{Consistency test}
The consistency test aims to observe a similar behaviour of the system when the
parameters are changed in such a way that the result is expected to be still the same.

With the hypothesis of First Come First Served scheduling policy and 
exponential distributions of times, the first tests verify the scenarios 
with fixed $\texttt{p}=0.5$ and $\texttt{N}=6$ and changing values of \texttt{E[T]} and \texttt{E[D]}, 
keeping the ratio $\frac{\texttt{E[D]}}{\texttt{E[T]}}$ still constant and equal to 10. In this way, the generation
rate and the service rate change in th same way and the statistics related to
the utilization should remain the same.In particular, the values chosen are 1ms, 
4ms and 10ms for \texttt{E[T]} and 10ms, 40ms and 100ms for \texttt{E[D]}. 
The simulation effectively shows the same CPU utilization around the value of
0.835. Also the mean number of busy CPUs is constant and equal to 5 and the mean
number of processes in the ready queue is around 4.7. The mean turnaround time, 
waiting time in the queue and service time show, instead, a proportionality
with $\texttt{E[D]}\propto \frac{1}{\mu}$ as expected. The times are multiplied 
with a factor 4 and 10 in the second and third scenarios with respect to the first one.

A second continuity test consists in changing also the number of CPUs \texttt{N},
changing the ratio $\frac{\texttt{E[D]}}{\texttt{E[T]}}$ but keeping constant
$\frac{\texttt{E[D]}}{\texttt{E[T]}\cdot \texttt{N}}$, proportional to the
utilization of the CPUs $\rho=\frac{\lambda}{N\cdot \mu}$. The chosen values
are 3, 6 and 9 for \texttt{N}, 1ms, 4ms and 10ms for \texttt{E[T]} and 
5ms, 40ms and 100ms for \texttt{E[D]}. The CPU utilization extracted from
the simulation effectively remains the same in the different scenarios and equal
to 0.835. The mean number of busy CPUs increases proportionally with
$\frac{\texttt{E[D]}}{\texttt{E[T]}}$: 2.5, 5 and 7.5 in the three cases. The other 
metrics change without a clear proportionality with the parameters, still consistently 
with the changes and with the expectations for a M/M/C system.

Another important test consists in changing these parameters in conditions
of instability. For example, in the latter case with \texttt{N} equal to 1, 2 and 3
the system keeps being unstable: the CPUs cannot handle the load of
the processes and the queue grows indefinitely.

The same tests have been performed combining the scenarios with the Shortest 
Job First queueing policy and with uniform distributions of the times. The results
confirm the consistency of the system with the expected behaviour in term of
CPU utilization and stability conditions. Of course there is no more the same
proportionality between the times and the metrics as regards queueing, while it 
is still valid for the service times of the processes.

\section{Continuity test}
The continuity test verifies that a slight change in the input does not 
cause wild changes on the system behaviour.

In the system, the test is handled in both FCFS and SJF scheduling policies.
The parameters vary with small steps of 5\% of their value, so that 
\texttt{N} takes values 19, 20 and 21, \texttt{p} takes values 0.475, 0.5 and 0.525,
\texttt{E[T]} takes values 1.425ms, 1.5ms and 1.575ms and \texttt{E[D]} takes values
47.5ms, 50ms and 52.5ms. The system is stable in all the cases, as expected,
and the variation of the metrics is around no more than 10\% of their value. For
example, the CPU utilization varies from a minimum of 0.817 to a maximum of 0.854 (4.33\% of variation).
The turnaround time goes from 49.7ms to 55.4ms (10.3\%), and similarly do the other
statistics.
% TODO: aggiungere grafico? CDF?



\chapter{Analysis}

\section{Calibration of the parameters}

Having a lot of parameters to calibrate, we decided to limit the number of possible values for each parameter. This choice was made to reduce the number of simulations to be performed and to have a more manageable number of results to analyze. The parameters that we decided to calibrate are the following:

\begin{itemize}
    \item \texttt{meanGenerationTime}: \{20ms, 50ms\} \\
          The mean for the generation times of the processes was chosen by considering a system under heavy and medium load.
    \item \texttt{meanProcessDuration}: \{40ms, 100ms, 200ms\} \\
          The mean for the duration of the processes was chosen after the estimations on the system stability. These values permit to have some combination of parameters that lead to a stable system and others that lead to an unstable system.
    \item \texttt{pCpuBound}: \{0.25, 0.75\} \\
          The probability of a process being CPU-bound has two values, one for a system with a high number of I-O bound processes and the other for a system with a high number of CPU-bound processes.
    \item \texttt{numCpus}: \{1, 4, 12\} \\
          The number of CPUs was chosen to simulate single-core, quad-core, and newer systems with 12 cores.
    \item \texttt{isFCFS}: \{true, false\} \\
          The scheduling policy that the scheduler uses can be either First-Come-First-Served or Shortest-Job-First.
    \item \texttt{generationType}: \{"exponential", "uniform"\} \\
          In addition to the exponential distribution, we also considered the uniform distribution for the generation of processes.
    \item \texttt{durationType}: \{"exponential", "uniform"\} \\
          In addition to the exponential distribution, we also considered the uniform distribution for the duration of the processes.
\end{itemize}


\section{Time parameters setup}

\subsection{Warm-up period}

To determine the warm-up period, we observed the development of the mean number of busy CPUs over time, through 10 independent runs and with different parameters configurations. The number of busy CPUs is a good indicator of the system's stability, as it doesn't depend on the scheduling policy.
Our observations showed that the system stabilizes well before \SI{200}{\second} in all tested configurations. To handle variability and ensure robustness in worse cases, we set the warm-up period to \SI{200}{\second}.

\begin{figure}[H]
    \captionsetup{type=figure}
    \centering
    \includegraphics[width=0.9\textwidth]{./images/04/lineWarmup.png}
    \captionof{figure}{Line chart of the mean number of busy CPUs over time for different configurations of meanGenerationTime and meanProcessDuration and multiple repetitions.}
    \label{fig:lineWarmup}
\end{figure}

\begin{figure}[H]
    \captionsetup{type=figure}
    \centering
    \includegraphics[width=0.9\textwidth]{./images/04/errorWarmup.png}
    \captionof{figure}{Average and Std dev of the mean number of busy CPUs over the repetitions.}
    \label{fig:errorWarmup}
\end{figure}


\begin{table}[H]
    \centering
    \begin{tabular}{c|cc|cc|cc|cc}
                 & \multicolumn{2}{c|}{20ms, 100ms} & \multicolumn{2}{c|}{20ms, 200ms} & \multicolumn{2}{c|}{50ms, 100ms} & \multicolumn{2}{c}{50ms, 200ms}                                   \\
        Time (s) & Avg                              & Std Dev                          & Avg                              & Std Dev                         & Avg  & Std Dev & Avg  & Std Dev \\
        \midrule
        1        & 1.52                             & 0.58                             & 2.54                             & 0.78                            & 0.68 & 0.46    & 1.20 & 0.80    \\
        2        & 1.54                             & 0.37                             & 2.85                             & 0.64                            & 0.67 & 0.27    & 1.25 & 0.51    \\
        5        & 1.59                             & 0.24                             & 3.10                             & 0.46                            & 0.63 & 0.15    & 1.24 & 0.29    \\
        10       & 1.63                             & 0.18                             & 3.18                             & 0.34                            & 0.65 & 0.10    & 1.29 & 0.20    \\
        20       & 1.68                             & 0.11                             & 3.33                             & 0.22                            & 0.64 & 0.08    & 1.27 & 0.15    \\
        50       & 1.72                             & 0.05                             & 3.43                             & 0.10                            & 0.68 & 0.05    & 1.35 & 0.09    \\
        100      & 1.74                             & 0.02                             & 3.47                             & 0.05                            & 0.69 & 0.02    & 1.37 & 0.04    \\
        200      & 1.75                             & 0.02                             & 3.49                             & 0.05                            & 0.69 & 0.02    & 1.38 & 0.03    \\
        500      & 1.75                             & 0.03                             & 3.50                             & 0.05                            & 0.70 & 0.01    & 1.40 & 0.02    \\
        1000     & 1.75                             & 0.01                             & 3.51                             & 0.02                            & 0.70 & 0.01    & 1.41 & 0.02    \\
    \end{tabular}
    \caption{Average and Std dev of the mean number of busy CPUs over the repetitions.}
    \label{tab:stabilization}
\end{table}

\subsection{Simulation duration}

For the simulation duration, we chose to end it at \SI{1000}{\second}. This ensures that after discarding the initial \SI{200}{\second} warm-up period, there remains \SI{800}{\second} of simulation data for analysis. Given that the mean generation time is on the order of tenths of seconds, this duration strikes a balance between obtaining statistically meaningful results even after subsampling and maintaining reasonable simulation times.


\section{Subsampling}

To analyze the data, the assumption of IID-ness will be needed, but without further modification the samples do not uphold it.
For instance, if a process finishes with a large turnaround time, which is caused by a long queue, it is likely that the same thing will happen for the next processes.

To address this and ensure independence between samples, subsampling has been employed. The new sample is constructed taking each point of the starting sample with probability $p$.
$p = \frac{1}{2^k}$ and $k$ is the smallest integer that passes the Ljung-Box test.
Using the Ljung-Box test makes it possible to automate the subsampling step. It was implmented to test the first 30 lags, with a significance level of 5\%, with these values the sample is considered independent if:
\vspace{-0.5\baselineskip}
\begin{equation}
    Q = n(n+2) \sum_{k=1}^{30} \frac{\hat{\rho}_k^2}{n-k} < \chi^2_{0.95,30} \approx 43.77
\end{equation}
The closer the system is to saturation, the stronger the correlation becomes as it can be seen in \cref{fig:autoCorComparison}.

% todo dare nomi ai grafici con metrica che ti dice quanto saturo (bello se definita nella parte iniziale)
\begin{figure}[H]
    \captionsetup{type=figure}
    \centering
    \begin{subfigure}[b]{0.45\textwidth}
        \includegraphics[width=\textwidth]{./images/04/autoCorHighUnfix.png}
        \caption{System close to saturation. Q = 1099}
        \label{fig:autoCorHighUnfix}
    \end{subfigure}
    \hfill % Add space between the subfigures
    \begin{subfigure}[b]{0.45\textwidth}
        \includegraphics[width=\textwidth]{./images/04/autoCorLowUnfix.png}
        \caption{System with high load. Q = 129}
        \label{fig:autoCorLowUnfix}
    \end{subfigure}
    
    \vspace{10pt} % Add vertical space between the two rows
    
    \begin{subfigure}[b]{0.45\textwidth}
        \includegraphics[width=\textwidth]{./images/04/autoCorHighFix.png}
        \caption{System close to saturation with $p=1/16$. Q = 25}
        \label{fig:autoCorHighFix}
    \end{subfigure}
    \hfill % Add space between the subfigures
    \begin{subfigure}[b]{0.45\textwidth}
        \includegraphics[width=\textwidth]{./images/04/autoCorLowFix.png}
        \caption{System with high load with $p=1/4$. Q = 20}
        \label{fig:autoCorLowFix}
    \end{subfigure}
    
    \vspace{10pt} % Add vertical space before the caption
    \caption{Comparison of turnaround time autocorrelation with and without subsampling for different loads.}
    \label{fig:autoCorComparison}
\end{figure}

\section{Statistical analysis}

For each of our measurements we made sure that the sample variance was stable for different simulation time, proving that the variance of the measurements is finite, and for each of them we collected more than 500 samples. Since after subsampling our observations are IID we can use the central limit theorem to state that:

% todo cambiare e usare direttamente la formula del CI

\begin{equation}
    Z = \frac{\overline{X} - \mu}{\frac{S}{\sqrt{n}}} \sim N(0,1)
\end{equation}

We chose to use a confidence interval of 95\% for our analysis.

\chapter{Conclusions}

% \section{Introduction Example}

\lipsum[1-3]

\begin{code}
    \mintedCode{cpp}{listings/example/test.cpp}
    \captionof{listing}{\texttt{test.cpp}}
    \label{code:test}
\end{code}

\lipsum[66]

\cuno{Commento 1}

\cdue{Commento 2}

Test cite \supercite{GOOGLE}.

Test footnote\footnote{Footnote}.

\begin{figure}[h]
    \captionsetup{type=figure}
    \centering
    \includegraphics[width=0.9\textwidth]{./images/example/gattino.png}
    \captionof{figure}{Un gattino.}
    \label{fig:gattino}
\end{figure}

Ciao

\lipsum[66]

% Cite sources that are not cited in the text
% \nocite{*}
\printbibliography

\end{document}

%----------------------------------------------------
