\chapter{Analysis}

\section{Calibration of the parameters}

Having a lot of parameters to calibrate, we decided to limit the number of possible values for each parameter. This choice was made to reduce the number of simulations to be performed and to have a more manageable number of results to analyze. The parameters that we decided to calibrate are the following:

\begin{itemize}
    \item \texttt{meanGenerationTime}: \{20ms, 50ms\} \\
    The mean for the generation times of the processes was chosen by considering a system under heavy and medium load.
    \item \texttt{meanProcessDuration}: \{40ms, 100ms, 200ms\} \\
    The mean for the duration of the processes was chosen after the estimations on the system stability. These values permit to have some combination of parameters that lead to a stable system and others that lead to an unstable system.
    \item \texttt{pCpuBound}: \{0.25, 0.75\} \\
    The probability of a process being CPU-bound has two values, one for a system with a high number of I-O bound processes and the other for a system with a high number of CPU-bound processes.
    \item \texttt{numCpus}: \{1, 4, 12\} \\
    The number of CPUs was chosen to simulate single-core, quad-core, and newer systems with 12 cores.
    \item \texttt{isFCFS}: \{true, false\} \\
    The scheduling policy that the scheduler uses can be either First-Come-First-Served or Shortest-Job-First.
    \item \texttt{generationType}: \{"exponential", "uniform"\} \\
    In addition to the exponential distribution, we also considered the uniform distribution for the generation of processes.
    \item \texttt{durationType}: \{"exponential", "uniform"\} \\
    In addition to the exponential distribution, we also considered the uniform distribution for the duration of the processes.
\end{itemize}


\section{Time parameters setup}

\subsection{Warm-up period}

To determine the warm-up period, we observed the development of the mean number of busy CPUs over time, through 10 independent runs and with different parameters configurations. The number of busy CPUs is a good indicator of the system's stability, as it doesn't depend on the scheduling policy.
We observed that the system stabilizes after \SI{200}{\second} in all of the above configurations and so we decided to set the warm-up period to \SI{200}{\second}.

\begin{figure}[H]
    \captionsetup{type=figure}
    \centering
    \includegraphics[width=0.9\textwidth]{./images/04/lineWarmup.png}
    \captionof{figure}{Line chart of the mean number of busy CPUs over time for different configurations of meanGenerationTime and meanProcessDuration and multiple repetitions.}
    \label{fig:lineWarmup}
\end{figure}
% todo
\begin{figure}[h]
    \captionsetup{type=figure}
    \centering
    \includegraphics[width=0.9\textwidth]{./images/04/errorWarmup.png}
    \captionof{figure}{Average and Std dev of the mean number of busy CPUs over the repetitions.}
    \label{fig:errorWarmup}
\end{figure}


\begin{table}[h]
\centering
\label{tab:performance}
\begin{tabular}{c|cc|cc|cc|cc}
 & \multicolumn{2}{c|}{20ms, 100ms} & \multicolumn{2}{c|}{20ms, 200ms} & \multicolumn{2}{c|}{50ms, 100ms} & \multicolumn{2}{c}{50ms, 200ms} \\
 Time (s) & Avg & Std Dev & Avg & Std Dev & Avg & Std Dev & Avg & Std Dev \\
\midrule
1 & 1.52 & 0.58 & 2.54 & 0.78 & 0.68 & 0.46 & 1.20 & 0.80 \\
2 & 1.54 & 0.37 & 2.85 & 0.64 & 0.67 & 0.27 & 1.25 & 0.51 \\
5 & 1.59 & 0.24 & 3.10 & 0.46 & 0.63 & 0.15 & 1.24 & 0.29 \\
10 & 1.63 & 0.18 & 3.18 & 0.34 & 0.65 & 0.10 & 1.29 & 0.20 \\
20 & 1.68 & 0.11 & 3.33 & 0.22 & 0.64 & 0.08 & 1.27 & 0.15 \\
50 & 1.72 & 0.05 & 3.43 & 0.10 & 0.68 & 0.05 & 1.35 & 0.09 \\
100 & 1.74 & 0.02 & 3.47 & 0.05 & 0.69 & 0.02 & 1.37 & 0.04 \\
200 & 1.75 & 0.02 & 3.49 & 0.05 & 0.69 & 0.02 & 1.38 & 0.03 \\
500 & 1.75 & 0.03 & 3.50 & 0.05 & 0.70 & 0.01 & 1.40 & 0.02 \\
1000 & 1.75 & 0.01 & 3.51 & 0.02 & 0.70 & 0.01 & 1.41 & 0.02 \\
\end{tabular}
\end{table}

    


\subsection{Simulation duration}