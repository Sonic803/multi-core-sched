\chapter{Verification}

\section{Code verification}
The code has been verified to work as expected with the debug GUI that comes 
with OMNeT++. In particular, the code has been tested trying to execute all the 
possible paths of the code in order to simulate all different cases, and comparing 
the actual state of the system with the expected one.

For the analysis, the specific constructs of OMNeT++ (such as cQueue for the queues 
or user defined messages) allowed to better follow the flow of events in the simulations.
In fact the inspection of the content of messages and the state of the queues has been 
crucial to understand the behaviour of the system.

Additional information has been gathered printing log messages in the console with the 
use of \texttt{EV}, in order to have a better comprehension of the flow of the simulation.
For example, \texttt{EV} has been very useful to track the main events of the 
simulation, such as the arrival of a new process, the start and end of the three phases 
of execution, the assignment of a process to a CPU. 
A simple graphical representation of the processes queued waiting to be assigned to a 
CPU helped to verify the correct behaviour of the scheduler in both FCFS and SJF policies.

\section{Degeneracy test}
Degeneracy tests aim to verify the correct behaviour of the system at extreme 
values of the parameters. The main quantitative parameters are the number of CPUs 
\texttt{N}, the probability of a process to be CPU bound \texttt{p}, the mean generation time
\texttt{E[T]} and the mean process duration \texttt{E[D]}. The possible scenarios
can be tested with FCFS or SJF scheduling policy and with exponential or uniform 
distributions of the two times.

The probability \texttt{p} set at 0 or 1 leads to the generation of
only I/O bound or CPU bound processes. The simulation works as expected in the 
two cases with different values of the statistics (ex. turnaround and waiting time, 
number of busy processors etc.), without affecting the functionality of the system.

When the number of CPUs is set to 0, no CPU is instantiated and the processes
are never assigned to a CPU. The system works as expected, with the processes
waiting in the queue until the finish of the simulation time or the end of
memory availability. Tests with only 1 CPU show a good behaviour and no problem
related to the handling of vectors of gates with length 1. The test with
a huge number of CPUs (100) is not different from every test with more than 
one CPU with respect to the multicore functionalities of the computer; obviously,
in this case, there is always an available CPU to assign a process to and 
there is no queueing.

Talking about the mean generation time \texttt{E[T]} and the mean process duration 
\texttt{E[D]}, what matters is the ratio between the two. To simulate scenarios
with very large (small) generation time and very small (large) process duration, the two
parameters have been set to $10^3$ms and $10^{-3}$ms respectively, and vice versa.
In the case of $\texttt{E[T]}>>\texttt{E[D]}$, every process executes its three phases 
before the generation of a new one, so there is no queueing as expected.
In the case of $\texttt{E[T]}<<\texttt{E[D]}$, the processes are generated at a very high rate
so it is necessary to set \texttt{sim-time-limit} to a small value to 
end the simulation in a reasonable time. The system works as expected, with
a high number of processes in the queue and CPUs always busy.

\section{Consistency test}
The consistency test aims to observe a similar behaviour of the system when the
parameters are changed in such a way that the result is expected to be still the same.

With the hypothesis of First Come First Served scheduling policy and 
exponential distributions of times, the first tests verify the scenarios 
with fixed $\texttt{p}=0.5$ and $\texttt{N}=6$ and changing values of \texttt{E[T]} and \texttt{E[D]}, 
keeping the ratio $\frac{\texttt{E[D]}}{\texttt{E[T]}}$ still constant and equal to 10. In this way, the generation
rate and the service rate change in th same way and the statistics related to
the utilization should remain the same.In particular, the values chosen are 1ms, 
4ms and 10ms for \texttt{E[T]} and 10ms, 40ms and 150ms for \texttt{E[D]}. 
The simulation effectively shows the same CPU utilization around the value of
0.835. Also the mean number of busy CPUs is constant and equal to 5 and the mean
number of processes in the ready queue is around 4.7. The mean turnaround time, 
waiting time in the queue and service time show, instead, a proportionality
with $\texttt{E[D]}\propto \frac{1}{\mu}$ as expected. The times are multiplied 
with a factor 4 and 10 in the second and third scenarios with respect to the first one.

A second consistency test consists in changing also the number of CPUs \texttt{N},
changing the ratio $\frac{\texttt{E[D]}}{\texttt{E[T]}}$ but keeping constant
$\frac{\texttt{E[D]}}{\texttt{E[T]}\cdot \texttt{N}}$, proportional to the
utilization of the CPUs $\rho=\frac{\lambda}{N\cdot \mu}$. The chosen values
are 3, 6 and 9 for \texttt{N}, 1ms, 4ms and 10ms for \texttt{E[T]} and 
5ms, 40ms and 100ms for \texttt{E[D]}. The CPU utilization extracted from
the simulation effectively remains the same in the different scenarios and equal
to 0.835. The mean number of busy CPUs increases proportionally with
$\frac{\texttt{E[D]}}{\texttt{E[T]}}$: 2.5, 5 and 7.5 in the three cases. The other 
metrics change without a clear proportionality with the parameters, still consistently 
with the changes and with the expectations for a M/M/C system.

Another important test consists in changing these parameters in conditions
of instability. For example, in the latter case with \texttt{N} equal to 1, 2 and 3
the system keeps being unstable: the CPUs cannot handle the load of
the processes and the queue grows indefinitely.

The same tests have been performed combining the scenarios with the Shortest 
Job First queueing policy and with uniform distributions of the times. The results
confirm the consistency of the system with the expected behaviour in term of
CPU utilization and stability conditions. Of course there is no more the same
proportionality between the times and the metrics as regards queueing, while it 
is still valid for the service times of the processes.

\section{Continuity test}
The continuity test verifies that a slight change in the input does not 
cause wild changes on the system behaviour.

In the system, the test is handled in both FCFS and SJF scheduling policies.
The parameters vary with small steps of 5\% of their value, so that 
\texttt{N} takes values 19, 20 and 21, \texttt{p} takes values 0.475, 0.5 and 0.525,
\texttt{E[T]} takes values 1.425ms, 1.5ms and 1.575ms and \texttt{E[D]} takes values
47.5ms, 50ms and 52.5ms. The system is stable in all the cases, as expected,
and the variation of the metrics is around no more than 10\% of their value. For
example, the CPU utilization varies from a minimum of 0.817 to a maximum of 0.854 (4.33\% of variation).
The turnaround time goes from 49.7ms to 55.4ms (10.3\%), and similarly do the other
statistics.
% TODO: aggiungere grafico? CDF?


