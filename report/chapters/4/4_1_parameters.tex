
\section{Calibration of the parameters}

Having a lot of parameters to calibrate, we decided to limit the number of possible values for each parameter. This choice was made to reduce the number of simulations to be performed and to have a more manageable number of results to analyze. The parameters that we decided to calibrate are the following:

\begin{itemize}
    \item \texttt{pCpuBound}: \{0.1, 0.9\} \\
          The probability of a process being CPU-bound has two values, one for a system with a high number of I-O bound processes and the other for a system with a high number of CPU-bound processes.
    \item \texttt{meanProcessDuration}: \{\SI{0.1}{\second}\} \\
          We chose a fixed value for the mean process duration, by doing this we can focus on the effect of the other parameters on the metrics.
    \item \texttt{meanGenerationTime}:\\
          The mean generation time was chosen to have a comparable load on the system for both values of \texttt{pCpuBound}. For each value of \texttt{pCpuBound} there is a simulation with heavy load and one with a lighter load. The values are in the order of tens of milliseconds.
    \item \texttt{numCpus}: \{4, 12\} \\
          The number of CPUs was chosen to simulate quad-core, and newer systems with 12 cores.
    \item \texttt{isFCFS}: \{true, false\} \\
          The scheduling policy that the scheduler uses can be either First-Come-First-Served or Shortest-Job-First.
    \item \texttt{generationType}: \{"exponential", "uniform"\} \\
          In addition to the exponential distribution, we also considered the uniform distribution for the generation of processes.
    \item \texttt{durationType}: \{"exponential", "uniform"\} \\
          In addition to the exponential distribution, we also considered the uniform distribution for the duration of the processes.
\end{itemize}
