
\section{Time parameters setup}

\subsection{Warm-up period}

To determine the warm-up period, we observed the development of the mean number of busy CPUs over time, through 10 independent runs and with different parameters configurations. The number of busy CPUs is a good indicator of the system's stability, as it doesn't depend on the scheduling policy.
Our observations showed that the system stabilizes well before \SI{200}{\second} in all tested configurations. To handle variability and ensure robustness in worse cases, we set the warm-up period to \SI{200}{\second}.

\begin{figure}[H]
    \captionsetup{type=figure}
    \centering
    \includegraphics[width=0.9\textwidth]{./images/04/lineWarmup.png}
    \captionof{figure}{Line chart of the mean number of busy CPUs over time for different configurations of meanGenerationTime and meanProcessDuration and multiple repetitions.}
    \label{fig:lineWarmup}
\end{figure}

\begin{figure}[H]
    \captionsetup{type=figure}
    \centering
    \includegraphics[width=0.9\textwidth]{./images/04/errorWarmup.png}
    \captionof{figure}{Average and Std dev of the mean number of busy CPUs over the repetitions.}
    \label{fig:errorWarmup}
\end{figure}


\begin{table}[H]
    \centering
    \scriptsize
    \begin{tabular}{c|cc|cc|cc|cc}
    & \multicolumn{2}{c|}{20ms, 100ms} & \multicolumn{2}{c|}{20ms, 200ms} & \multicolumn{2}{c|}{50ms, 100ms} & \multicolumn{2}{c}{50ms, 200ms}                                   \\
Time (s) & Avg                              & Std Dev                          & Avg                              & Std Dev                         & Avg  & Std Dev & Avg  & Std Dev \\
\midrule
1        & 1.52                             & 0.58                             & 2.54                             & 0.78                            & 0.68 & 0.46    & 1.20 & 0.80    \\
2        & 1.54                             & 0.37                             & 2.85                             & 0.64                            & 0.67 & 0.27    & 1.25 & 0.51    \\
5        & 1.59                             & 0.24                             & 3.10                             & 0.46                            & 0.63 & 0.15    & 1.24 & 0.29    \\
10       & 1.63                             & 0.18                             & 3.18                             & 0.34                            & 0.65 & 0.10    & 1.29 & 0.20    \\
20       & 1.68                             & 0.11                             & 3.33                             & 0.22                            & 0.64 & 0.08    & 1.27 & 0.15    \\
50       & 1.72                             & 0.05                             & 3.43                             & 0.10                            & 0.68 & 0.05    & 1.35 & 0.09    \\
100      & 1.74                             & 0.02                             & 3.47                             & 0.05                            & 0.69 & 0.02    & 1.37 & 0.04    \\
200      & 1.75                             & 0.02                             & 3.49                             & 0.05                            & 0.69 & 0.02    & 1.38 & 0.03    \\
500      & 1.75                             & 0.03                             & 3.50                             & 0.05                            & 0.70 & 0.01    & 1.40 & 0.02    \\
1000     & 1.75                             & 0.01                             & 3.51                             & 0.02                            & 0.70 & 0.01    & 1.41 & 0.02    \\
\end{tabular}
    \caption{Average and Std dev of the mean number of busy CPUs over the repetitions.}
    \label{tab:stabilization}
\end{table}

\subsection{Simulation duration}

For the simulation duration, we chose to end it at \SI{1000}{\second}. This ensures that after discarding the initial \SI{200}{\second} warm-up period, there remains \SI{800}{\second} of simulation data for analysis. Given that the mean generation time is on the order of tenths of a second, this duration strikes a balance between obtaining statistically meaningful results even after subsampling and maintaining reasonable simulation times.
