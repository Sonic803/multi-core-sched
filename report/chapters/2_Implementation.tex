\chapter{Implementation}
We implemented the Computer system in OMNeT++.\\
Figure \autoref{fig:omnetpp_implementation} provides an overview of the system implementation. Below is a detailed explanation of each module:
\begin{itemize}
    \item \texttt{ProcessGenerator}: this module generates processes according to a specified interarrival time distribution. Each process is assigned a total duration $D$, and the duration of each phase (initial, I/O, and final) is set to $0.4 D$, $0.2 D$, and $0.4 D$ if the process is CPU bound, or $0.1 D$, $0.8 D$, and $0.1 D$ if the process is I/O bound. The process is then sent to the scheduler.
    \item \texttt{scheduler}: this module contains an infinite-capacity queue. If \texttt{isFCFS} is set to \texttt{true}, a FIFO queue is used. Otherwise, a priority queue is used, with processes ordered by increasing duration until the I/O phase or termination (shortest job first).\\
    When a new process arrives, the scheduler sends it to the first free CPU. If no CPU is free, the process is enqueued. When a CPU is freed, the next process in the queue is sent.\\
    When a process enters the I/O phase, the scheduler waits for the specified I/O phase duration and adds the process back into the queue.\\
    When a process terminates, its turnaround time and waiting time are emitted as signals.\\
    The scheduler also keeps track of the number of active CPUs and the queue length, emitting those every time a message is received.

    \item \texttt{cpu}: There are $N$ CPUs which operate independently. Each one, when it is free, receives a process from the scheduler and simulates its execution. When it finishes the execution phase (i.e. the I/O phase is reached or the process terminates), a message is sent to the scheduler, indicating that the CPU is now free. Every CPU records its utilization, emitting the busy time as a signal when the CPU is freed.
\end{itemize}

\begin{figure}[H]
    \captionsetup{type=figure}
    \centering
    \includegraphics[width=0.4\textwidth]{images/example/Computer.pdf}
    \captionof{figure}{Overview of the system implementation.}
    \label{fig:omnetpp_implementation}
\end{figure}
